\documentclass{scrreprt}
\hyphenpenalty=1000000
\usepackage{listings}
\usepackage{underscore}
\usepackage{graphicx}
\usepackage[bookmarks=true]{hyperref}
\usepackage[utf8]{inputenc}
\usepackage[english]{babel}
\usepackage{enumitem}
\usepackage{datetime2}

%\usepackage{draftwatermark}
%\SetWatermarkText{FOR 2ND REVIEW}
%\SetWatermarkScale{1}

\hypersetup{
    bookmarks=false,
    pdftitle={Software Requirement Specification Team 8},
    pdfauthor={Team 8},
    pdfsubject={SRS for Train Control System},
    pdfkeywords={TeX, LaTeX, graphics, images},
    colorlinks=true,
    linkcolor=blue,
    citecolor=black,
    filecolor=black,
    urlcolor=purple,
    linktoc=page
}

\def\myversion{1.1}
\date{}

\usepackage{hyperref}

\linespread{1.25}
\begin{document}

\begin{flushright}
    \rule{15cm}{5pt}\vskip1cm
    \begin{bfseries}
        \Huge{SOFTWARE REQUIREMENTS\\ SPECIFICATION}\\
        \vspace{1.5cm}
        for\\
        \vspace{1.5cm}
        Train Control System\\
        \vspace{1.5cm}
        \LARGE{Version \myversion}\\
        \vspace{1.5cm}
        Prepared by: Team 8\\
        \small{Andrew White\ \ \ \ \ Sara Keriakes \\
        Grace Keywork\ \ \ \ \ Tim Bottegal \\
        Jay Fu\ \ \ \ Johann Traum \\
        \vspace{1.0cm}}
        \large{
        \today
        }\\
    \end{bfseries}
\end{flushright}


\tableofcontents

\chapter{Introduction}

    \section{Purpose}
        This document will outline all software requirements for the Train Control System (TCS). It will provide an overview of the final deliverable and will specify both functional and non-functional requirements.

        This document will also serve as the contributors' main tool of clarifying project requirements with users and stakeholders.

    \section{Scope}
        This Software Requirements Specification (SRS) defines the scope of the Train Control System (TCS), a system designed to manage and control train operations across railway networks. The system is comprised of several modules: The Centralized Track Control (CTC) Office, The Track Controller (Wayside Controller), Track Model, Train Model, and Train Controller.

        The primary goal of the TCS is to ensure passenger safety while facilitating the efficient and reliable movement of trains while conforming to the needs and expectations of our users and other stakeholders. Key users of the TCS include the Programmer, Dispatcher, Driver, Passenger, Murphy, Track Builder, and Engineer. 

    \section{Definitions, Acronyms, \& Abbreviations}
        \begin{enumerate}
            \item \textbf{SRS}: Software Requirements Specification
            \item \textbf{TCS}: Train Control System
            \item \textbf{PLC}: Programmable Logic Controller
            \item \textbf{RFID}: Radio Frequency Identification
            \item \textbf{UI}: User Interface
            \item $\mathbf{K_i}$\textbf{:} Integral Gain
            \item $\mathbf{K_p}$\textbf{:} Proportional Gain
            \item \textbf{Vital}: Processes that are critical for maintaining safe operation of the Train Control System
            \item \textbf{CTC Office}: Centralized Traffic Control Office
            \item \textbf{Wayside Controller}: Track Controller
            \item \textbf{Contributors}: Team 8 members responsible for the design and implementation of the Train Control System.
            \item \textbf{Murphy}: User whose main goal is to cause harm to the Train Control System
        \end{enumerate}

    \section{References}
        \textit{There are no references to outside material at this time.}

    \section{Overview}
        This SRS is divided into two sections: Overall Description and Specific Requirements. 
        The Overall Description section provides a high-level overview of the TCS, outlining its functions. This section also discusses the users of the system, outside constraints to the design of the system, and the Contributors' ethos on the system's design.
        The Specific Requirements discusses the system's functions and requirements at a low level, both functional and non-functional in nature. It lists the granular requirements of each module in easily testable statements.

\chapter{Overall Description}

    \section{Product Perspective}
        This system is a self-contained train simulation environment. The key modules of the simulation environment include a CTC Office interface, Track Controller software interface, Track Controller hardware interface, a Track Model, a Train Model, a Train Controller software interface, and a Train Controller hardware interface. These modules will then be integrated and work together to perform the overall simulation. 

    \section{Product Functions}
        The system will be able to safely and automatically guide trains along transit lines from an uploaded schedule, as well as accept manual input from users.

        \subsubsection{Per-Module Features}
        \begin{itemize} 

            \item \textbf{Track Model:}
                Maintain a flexible model of the railway layout incorporating blocks, signals, switches, and stations. 
                       
        \end{itemize}

\chapter{System Architecture}

    \section{External Interfaces}
        
        \subsubsection{Track Model}
        \begin{itemize}
            \item \textbf{Inputs:}
                \begin{enumerate}
                    \item From the Track Controller: Switch commands, crossing gate commands, signal commands (various python functions)
                    \item From the Train Model: Block occupancy (bool)
                \end{enumerate}
            \item \textbf{Outputs:}
                \begin{enumerate}
                    \item To the Track Controller: Status of switch positions (bool), Signal states (bool), Block occupancy (bool), Broken rail detection (bool), Crossing status (bool)
                    \item To the Train Model: Percent Grade (\%), Beacon information (string), Number of persons boarding at a station (int)
                \end{enumerate}
        \end{itemize}

    \section{Functional Requirements}
            
        \subsubsection{Track Model}
            \begin{enumerate}
                \item The Track Model shall load a model (representation) of the track layout.
                \item The Track Model shall keep a log of ticket sales.
                \item The Track Model shall keep a log of the number of passengers entering and leaving the station.
                \item The Track Model shall display block occupancy, switch positions, and light states to the Track Builder.
                \item The Track Model representation shall have an environmental temperature set by Murphy.
                \item The Track Model shall have track heaters that activate below a certain environmental temperature.
                \item The Track Model shall have the following failure states that are activated by Murphy: Broken Rail, Track Circuit Failure, and Power Failure.
            \end{enumerate}

    
    \section{Non-Functional Requirements}
        \subsection{Performance}
                
            \subsubsection{Track Model}
                \begin{enumerate}
                    \item The Track Model shall only have one terminal.
                    \item The Track Model shall support two users: Track Builder and Murphy.
                    \item The Track Model shall take the commanded speed and authority from the Track Controller and transmit them to the Train Model via track circuit.
                \end{enumerate}

        \subsection{Reliability}
                
            \subsubsection{Track Model}
                \begin{enumerate}
                    \item The system shall immediately report detected failures to the necessary module(s).
                    \item The Track Model shall periodically compare heater status to environmental temperature to ensure proper heater function.
                    \item The Track Model shall accurately display all track properties with correct units and consistency. 
                \end{enumerate}

        \subsection{Availability}
                
            \subsubsection{Track Model}
                \begin{enumerate}
                    \item The Track Model shall always make the track status parameters available to the Track Builder.
                    \item In the case of a software failure, the Track Model shall restart without prompt to the last valid state.
                \end{enumerate}
                
        \subsection{Safety}
                
            \subsubsection{Track Model}
                \begin{enumerate}
                    \item The Track Model shall maintain safe track conditions during adverse weather via the use of track heaters.
                    \item The Track Model shall detect the following failure conditions: Broken Rail, Track Circuit Failure, and Power Failure.
                \end{enumerate}
                
            \subsubsection{Train Model}
                \begin{enumerate}
                    \item The Train Model shall prevent its doors from opening while the train is moving.
                    \item The Train Model shall detect the following failure conditions: Engine Failure, Signal Pickup Failure, and Brake Failure. These failure detections shall then be relayed to the Train Controller for action.
                    \item The Train Model shall have an Emergency Brake operable by passengers in case of emergency.
                \end{enumerate}

        \subsection{Maintainability}
            \begin{enumerate}
                \item The system shall be divided into modules.
                \item The failure of one module shall not affect the functionality of other modules, with the exception of pass through data.
                \item Contributors shall attend regular scrum meetings to maintain system cohesiveness.
                \item Contributors shall follow the set Coding Standard, and proper GitHub commit and push practices shall be followed.
                
            \end{enumerate}

        \subsection{Portability}
            \begin{enumerate}
                \item The entirety of the system shall run on Windows 10 from a single executable.
            \end{enumerate}


    \section{Design Constraints}
        \begin{enumerate}
            \item The system simulation shall run on a 2016 Microsoft Surface Pro running Windows 10 22H2.
            \item Any module UI shall not be full screen to accommodate other module UI.
        \end{enumerate}

    \section{Additional Requirements}
        \begin{enumerate}
            \item The system shall be capable of running at least 10 times faster than wall clock time.
            \item The system shall be able to be paused.
        \end{enumerate}

%\chapter{Appendices}

    
    



\end{document}